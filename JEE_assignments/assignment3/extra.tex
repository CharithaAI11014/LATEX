%iffalse
\let\negmedspace\undefined
\let\negthickspace\undefined
\documentclass[journal,12pt,onecolumn]{IEEEtran}
\usepackage{cite}
\usepackage{amsmath,amssymb,amsfonts,amsthm}
\usepackage{algorithmic}
\usepackage{graphicx}
\usepackage{textcomp}
\usepackage{xcolor}
\usepackage{txfonts}
\usepackage{listings}
\usepackage{enumitem}
\usepackage{mathtools}
\usepackage{gensymb}
\usepackage{comment}
\usepackage[breaklinks=true]{hyperref}
\usepackage{tkz-euclide} 
\usepackage{listings}
\usepackage{gvv}      
%\def\inputGnumericTable{}                                 
\usepackage[latin1]{inputenc}                                
\usepackage{color}                                            
\usepackage{array}                                            
\usepackage{longtable}                                       
\usepackage{calc}                                             
\usepackage{multirow}    
\usepackage{hhline}                                           
\usepackage{ifthen}     

\usepackage{lscape}
\usepackage{tabularx}
\usepackage{array}
\usepackage{float}
\usepackage{multicol}

\newtheorem{theorem}{Theorem}[section]
\newtheorem{problem}{Problem}
\newtheorem{proposition}{Proposition}[section]
\newtheorem{lemma}{Lemma}[section]
\newtheorem{corollary}[theorem]{Corollary}
\newtheorem{example}{Example}[section]
\newtheorem{definition}[problem]{Definition}
\newcommand{\BEQA}{\begin{eqnarray}}
\newcommand{\EEQA}{\end{eqnarray}}
\newcommand{\define}{\stackrel{\triangle}{=}}
\theoremstyle{remark}
\newtheorem{rem}{Remark}

% Marks the beginning of the document
\begin{document}
\bibliographystyle{IEEEtran}
\vspace{3cm}

\title{21.PROBABILITY}
\author{ai24btech11014 - Charitha Sri}

\maketitle
\bigskip       
\renewcommand{\thefigure}{\theenumi}
\renewcommand{\thetable}{\theenumi}

\section{JEE Main / AIEEE}

\begin{enumerate}
	
	\item It is given that the events $A$ and $B$ are such that $\pr{A}=\frac{1}{4}, \pr{A | B}=\frac{1}{2}$ and $\pr{B | A}=\frac{2}{3}$. Then $\pr{B}$ is \hfill{\sbrak{2008}}
	\begin{enumerate}
	\begin{multicols}{4}
	\item $\frac{1}{6}$
	\item $\frac{1}{3}$
	\item $\frac{2}{3}$
	\item $\frac{1}{2}$
	\end{multicols}
        \end{enumerate}

\item A die is thrown. Let $A$ be the event that the number obtained is greater than 3. Let $B$ be the event that the number obtained is less than 5. Then $\pr{A\cup B}$ is \hfill{\sbrak{2008}}
	\begin{enumerate}
	\begin{multicols}{4}
	\item $\frac{3}{5}$
	\item $0$
	\item $1$
	\item $\frac{2}{5}$
	\end{multicols}
	\end{enumerate}

        \item In a binomial distribution $B\brak{n,p=\frac{1}{4}}$, if the probability of at least one success is greater than or equal to $\frac{9}{10}$, then n is greater than: \hfill{\sbrak{2009}}
	\begin{enumerate}
	\begin{multicols}{2}
	\item $\frac{1}{\log_{10}4+\log_{10}3}$
	\item $\frac{9}{\log_{10}4-\log_{10}3}$
	\item $\frac{4}{\log_{10}4-\log_{10}3}$
	\item $\frac{1}{\log_{10}4-\log_{10}3}$
	\end{multicols}
	\end{enumerate}

	\item One ticket is selected at random from 50 tickets numbered $00,01,02,\dots,49$. Then the probability that the sum of the digits on the selected ticket is 8, given that the product of these digits is zero, equals:
	\hfill{\sbrak{2009}}
	\begin{enumerate}
	\begin{multicols}{4}
	\item $\frac{1}{7}$
	\item $\frac{5}{14}$
	\item $\frac{1}{50}$
	\item $\frac{1}{14}$
	\end{multicols}
	\end{enumerate}

	\item Four numbers are chosen at random (without replacement) from the set $\cbrak{1,2,3,\dots 20}$. 
		$\\$
		\textbf{Statement-1:} The probability that the chosen numbers when arranged in some order will form an AP is $\frac{1}{85}$.
		$\\$
		\textbf{Statement-2:} If the four chosen numbers form an A.P, then the set of all possible values of common difference is $\brak{\pm 1,\pm 2,\pm 3,\pm 4,\pm 5}$.
		\hfill{\sbrak{2010}}
		\begin{enumerate}
		\item Statement-1 is true, Statement-2 is true; 
		Statement-2 is not a correct explanation for Statement-1
		\item Statement-1 is true, Statement-2 is false
		\item Statement-1 is false, Statement-2 is true.
		\item Statement-1 is true, Statement-2 is true; Statement-2 is a correct explanation for Statement-1.
		\end{enumerate}

	\item An urn contains nine balls of which three are red, four are blue and two are green. Three balls are drawn at random without replacement from the urn. The probability that the three balls have different colours is
		\hfill{\sbrak{2010}}
		\begin{enumerate}
		\begin{multicols}{4}
		\item $\frac{2}{7}$
		\item $\frac{1}{21}$
		\item $\frac{2}{23}$
		\item $\frac{1}{3}$
		\end{multicols}
		\end{enumerate}

		\item Consider 5 independent Bernoulli's trails each with probability of success p. If the probability of at least one failure is greater than or equal to $\frac{31}{32}$, then p lies in the interval
		\hfill{\sbrak{2011}}
		\begin{enumerate}
		\begin{multicols}{2}
		\item $\biggl(\frac{3}{4},\frac{11}{12}\biggr]$
		\item $\sbrak{0,\frac{1}{2}}$
		\item $\biggl(\frac{11}{12},1\biggr]$
		\item $\biggl(\frac{1}{2},\frac{3}{4}\biggr]$
		\end{multicols}
		\end{enumerate}

	\item If $C$ and $D$ are two events such that $C \subset D$ and $\pr{D}\neq0$. then the correct statement among the following is 
		\hfill{\sbrak{2011}}
		\begin{enumerate}
		\begin{multicols}{2}
		\item $\pr{C | D} \geq \pr{C}$
		\item $\pr{C | D}<\pr{C}$
		\item $\pr{C | D}=\frac{\pr{D}}{\pr{C}}$
		\item $\pr{C | D}=\pr{c}$
		\end{multicols}
		\end{enumerate}

	\item Three numbersare chosen at random without replacement from $\cbrak{1,2,3\dots8}$. The probability that their minimum is 3, given that their maximum is 6, is :
		\hfill{\sbrak{2012}}
		\begin{enumerate}
		\begin{multicols}{4}
		\item $\frac{3}{8}$
		\item $\frac{1}{5}$
		\item $\frac{1}{4}$
		\item $\frac{2}{5}$
		\end{multicols}
		\end{enumerate}

	\item A multiple choice examination has 5 questions. Each question has three alternative answers of which exactly one is correct. The probability that a student will get 4 or more correct answers just by guessing is:
		\hfill{\sbrak{JEE M 2013}}
		\begin{enumerate}
		\begin{multicols}{4}
		\item $\frac{17}{{3}^{5}}$
		\item $\frac{13}{{3}^{5}}$
		\item $\frac{11}{{3}^{5}}$
		\item $\frac{10}{{3}^{5}}$
		\end{multicols}
		\end{enumerate}

	\item Let $A$ and $B$ be two events such that $\pr{\overline{A\cup B}}=\frac{1}{6}$, $\pr{\overline{A \cap B}}=\frac{1}{4}$, and $\pr{\overline{A}}=\frac{1}{4}$, where $\overline{A}$ stands for the complement of the event A. Then the events $A$ and $B$ are
		\hfill{\sbrak{JEE M 2014}}
		\begin{enumerate}
		\begin{multicols}{2}
		\item independent but not equally likely.
		\item independent and equals likely.
		\item mutually exclusive and independent.
		\item equally likely but not independent.
		\end{multicols}
		\end{enumerate}

	\item If 12 identical balls are to be placed in 3 identical boxes, then the probability that one of the boxes contains exactly 3 balls is:
		\hfill{\sbrak{JEE M 2015}}
		\begin{enumerate}
		\begin{multicols}{2}
		\item $220\brak{\frac{1}{3}}^{12}$
		\item $22\brak{\frac{1}{3}}^{11}$
		\item $\frac{55}{3}\brak{\frac{2}{3}}^{11}$
		\item $55\brak{\frac{2}{3}}^{10}$
		\end{multicols}
		\end{enumerate}

	\item Let two fair six-faced dice $A$ and $B$ be the thrown simultaneously. If $E_{1}$ is the event thta die $A$ shows up four, $E_{2}$ is the event that die $B$ shows up two and $E_{3}$ is the event that the sum of numbers on both dice is odd, then which of the following statement is NOT true?
		\hfill{\sbrak{JEE M 2016}}
		\begin{enumerate}
		\begin{multicols}{2}
		\item $E_{1}$ and $E_{3}$ are independent.
		\item $E_{1}$, $E_{2}$ and $E_{3}$ are independent.
		\item $E_{1}$ and $E_{2}$ are independent.
		\item $E_{2}$ and $E_{3}$ are independent.
		\end{multicols}
		\end{enumerate}

	\item A box contains 15 green and 10 yellow balls. If 10 balls are randomly drawn, one-by-one, with replacement, then the variance of the number of green balls drawn is:
		\hfill{\sbrak{JEE M 2017}}
		\begin{enumerate}
		\begin{multicols}{4}
		\item $\frac{6}{25}$
		\item $\frac{12}{5}$
		\item 6
		\item 4
		\end{multicols}
		\end{enumerate}
	
	\item If two different numbers are taken from the set from the set $\brak{0,1,2,3,\dots,10}$. Then the probability that their sum as well as absolute difference are both multiple of 4, is:
		\hfill{\sbrak{JEE M 2017}}
		\begin{enumerate}
		\begin{multicols}{4}
		\item $\frac{7}{55}$
		\item $\frac{6}{55}$
		\item $\frac{12}{55}$
		\item $\frac{14}{55}$
		\end{multicols}
		\end{enumerate}
\end{enumerate}

\end{document}
