%iffalse
\let\negmedspace\undefined
\let\negthickspace\undefined
\documentclass[journal,12pt,onecolumn]{IEEEtran}
\usepackage{cite}
\usepackage{amsmath,amssymb,amsfonts,amsthm}
\usepackage{algorithmic}
\usepackage{graphicx}
\usepackage{textcomp}
\usepackage{xcolor}
\usepackage{txfonts}
\usepackage{listings}
\usepackage{enumitem}
\usepackage{mathtools}
\usepackage{gensymb}
\usepackage{comment}
\usepackage[breaklinks=true]{hyperref}
\usepackage{tkz-euclide} 
\usepackage{listings}
\usepackage{gvv}      
%\def\inputGnumericTable{}                                 
\usepackage[latin1]{inputenc}                                
\usepackage{color}                                            
\usepackage{array}                                            
\usepackage{longtable}                                       
\usepackage{calc}                                             
\usepackage{multirow}    
\usepackage{hhline}                                           
\usepackage{ifthen}     

\usepackage{lscape}
\usepackage{tabularx}
\usepackage{array}
\usepackage{float}
\usepackage{multicol}

\newtheorem{theorem}{Theorem}[section]
\newtheorem{problem}{Problem}
\newtheorem{proposition}{Proposition}[section]
\newtheorem{lemma}{Lemma}[section]
\newtheorem{corollary}[theorem]{Corollary}
\newtheorem{example}{Example}[section]
\newtheorem{definition}[problem]{Definition}
\newcommand{\BEQA}{\begin{eqnarray}}
\newcommand{\EEQA}{\end{eqnarray}}
\newcommand{\define}{\stackrel{\triangle}{=}}
\theoremstyle{remark}
\newtheorem{rem}{Remark}

% Marks the beginning of the document
\begin{document}
\bibliographystyle{IEEEtran}
\vspace{3cm}

\title{18.Definite Integrals}
\author{ai24btech11014 - Charitha Sri}

\maketitle
\bigskip       
\renewcommand{\thefigure}{\theenumi}
\renewcommand{\thetable}{\theenumi}
\section{G:Passage 6}



\begin{enumerate}
 \item Let $F: R\rightarrow R$ be a thrice differentiable function. Suppose that $F\brak{1}=0, F\brak{3}=-4$ and $F\brak{x}<0$ for all $x \in \brak{\frac{1}{2},3}$. Let $f\brak{x}=x F\brak{x}$ for all $ x \in \mathbf{R}$                        \hfill{(JEE Adv. 2015)}
\item The correct statement(s) is(are) 
 \begin{enumerate}
 \begin{multicols}{2}
 \item $f^\prime\brak{1}<0$
 \item $f\brak{2}<0$
 \item $f^\prime\brak{x}\neq0$ for any $x \in\brak{1,3}$
 \item $f^\prime\brak{x}=0$ for some $x \in\brak{1,3}$
 \end{multicols}
 \end{enumerate}

\item If $\int_{1}^{3}x^2F^\prime\brak{x}dx=-12$ and $\int_{1}^{3}x^3F"\brak{x}dx=40$ , then the correct expression(s) is(are) 
 \begin{enumerate}
 \begin{multicols}{2}
 \item $9f^\prime\brak{3}+f^\prime\brak{1}-32=0$
        \item $\int_{1}^{3}f\brak{x}dx=12$
        \item $9f^\prime(3)-f^\prime\brak{1}+32=0$
        \item $\int_{1}^{3}f\brak{x}dx=-12$
        \end{multicols}
        \end{enumerate}
\end{enumerate}

\section{Integer}
\begin{enumerate}

\item Let $f$ : $R\rightarrow R$ be a continuous function which 
 satisfies $f(x)=\int_{0}^{x}f\brak{t}dt.$ Then the value of $f\brak{ln5}$ is \hfill{(2009)}
  
         \item For any real number $x$, let \sbrak{x} denote the largest integer less than or equal to $x$. Let $f$ be a real valued function defined on the interval $\sbrak{-10,10}$ by
		 $f\brak{x}$= \[\begin{cases}  x-\sbrak{x} if \sbrak{x} is odd \\ 1+\sbrak{x}-x if \sbrak{x} is even \end{cases}\]
  Then the value of $\frac{{\pi}^{2}}{10}\int\limits_{-10}^{10}$ $f\brak{x}\cos\pi xdx$ is \hfill{(2011)}


         \item The value of 
		 \begin{align}
			 \int_{0}^{1} 4x^3 \cbrak{\frac{d^2}{dx^2}\brak{1-x^2}^5 }dx 
		 \end{align}
		 is \hfill{(JEE Adv. 2014)}

  \item Let $f$ : $R \rightarrow R$ be a function defined by
	  \begin{align}
		  f\brak{x}=\cbrak{\sbrak{x},x \leq2, 0 if x>2 }
	  \end{align}
  where $\sbrak{x}$ is the greatest integer less than or equal to $x$, if 
  \[ I=\int_{-1}^{2} \frac{xf\brak{x^2}}{2+f\brak{x+1}}dx \],
		then the value of $\brak{4I-1}$ is \hfill{(JEE Adv. 2015)}  

\item Let 
	\begin{align}
		F\brak{x}=\int_{x}^{x^2 +\frac{\pi}{6}} 2\cos^2t(dt)
	\end{align}
 for all $x \in$ R and $f:\sbrak{0,\frac{1}{2}}\rightarrow[0,\infty)$ be a continuous function. For $a \in\sbrak{0,\frac{1}{2}}$, if $F^\prime\brak{a}+2$ is the area of the region bounded by $x=0, y=0$, $y=f\brak{x}$ and $x=a$, then $f\brak{0}$ is   \hfill{(JEE Adv. 2015)}

\item If 
	\begin{align}
		\alpha=\int_{0}^{1}\brak{e^{9x+3\tan^{-1}x} } \ \brak{ \frac{12+9x^2}{1+x^2}}dx
	\end{align}
		where $tan^{-1}x$ takes only principal values,
 then the value of $\brak{\log_e |1+\alpha|-\frac{3\pi}{4}}$ is  \hfill{(JEE Adv. 2015)}

\item Let $f$ : $R \rightarrow R$ be a continuous odd function which vanishes exactly at one point and $f(1)=\frac{1}{2}$. Suppose that $F\brak{x}=\int_{-1}^{x}f\brak{t}dt$ for all $x\in\sbrak{-1,2}$ and $G\brak{x}=\int_{-1}^{x}t|f\brak{f\brak{t}}|dt$ for all $x\in \sbrak{-1,2}$. If 
	\begin{align}
		lim_{x\to 1} \frac{F\brak{x}}{G\brak{x}}=\frac{1}{14}
	\end{align}
		, then the value of $f\brak{\frac{1}{2}}$ is  \hfill{(JEE Adv. 2015)}

\item The total number of distinct $x\in\sbrak{0,1}$ for which 
	\begin{align}
		\int_{0}^{x}\frac{t^2}{1+t^4}dt=2x-1
	\end{align} is \hfill{(JEE Adv. 2016)}

\item Let $f:R\rightarrow$ R be a differentiable function such that $f\brak{0}=0$, $f\brak{\frac{\pi}{2}}=3$ and $f^\prime\brak{0}=1$. If 
	\begin{align}
		g\brak{x}=\int_{x}^{\frac{\pi}{2}}\sbrak{f^\prime\brak{t}\cosec t -\cot t\cosec tf\brak{t}}dt
	\end{align}
		for $x\in \bigg(0,\frac{\pi}{2}\biggr]$, then $\lim_{x\to 0} g\brak{x}= $   \hfill{(JEE Adv. 2018)}

\item For each positive integer n, let 
	\begin{align}
		y_n =\frac{1}{n}\brak{n+1}\brak{n+2}\dots\brak{n+n}^{\frac{1}{n}}
	\end{align}
		For $x \in R$, let \sbrak{x} be the greatest integer less than  or equal to x. If $\lim_{n\to \infty} y_n$=L, then the value of [L] is \hfill{(JEE Adv. 2018)}

\item A farmer $F_1$ has a land in the shape of a triangle with vertices at $\vec{P}=\brak{0,0}$,$\vec{Q}=\brak{1,1}$ and $\vec{R}=\brak{2,0}$. From this land, a neighbouring farmer $F_2$ takes away the region which lies between the side PQ and a curve of the form $y=x^n\brak{n>1}$. If the area of the region taken away by the farmer $F_2$ is exactly 30\% of the area of $\Delta$ PQR, then the value of n is \hfill{(JEE Adv. 2018)} 

\item The value of the integral \begin{align}
		\int_{0}^{\frac{1}{2}}\frac{1+\sqrt{3}}{\brak{\brak{x+1}^2\brak{1-x}^6}^\frac{1}{4}}dx
\end{align} \hfill{(JEE Adv. 2018)}

\item If  \begin{align} I=\frac{2}{\pi} \int_{\frac{-\pi}{4}}^{\frac{\pi}{4}} \frac{dx}{\brak{1+e^{\sin x}}\brak{2-\cos2x}},\end{align} then $27I^2$ equals \hfill{(JEE Adv. 2019)}

\item The value of the integral 
	\begin{align} \int_{0}^{\frac{\pi}{2}} \frac{3\sqrt{\cos\theta}}{\brak{\sqrt{\cos\theta}+\sqrt{\sin\theta}}^5} d\theta \text{equals} \end{align} \hfill{(JEE Adv. 2019)}
\end{enumerate}
\end{document}
