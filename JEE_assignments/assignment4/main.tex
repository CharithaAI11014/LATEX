%iffalse
\let\negmedspace\undefined
\let\negthickspace\undefined
\documentclass[journal,12pt,onecolumn]{IEEEtran}
\usepackage{cite}
\usepackage{amsmath,amssymb,amsfonts,amsthm}
\usepackage{algorithmic}
\usepackage{graphicx}
\usepackage{textcomp}
\usepackage{xcolor}
\usepackage{txfonts}
\usepackage{listings}
\usepackage{enumitem}
\usepackage{mathtools}
\usepackage{gensymb}
\usepackage{comment}
\usepackage[breaklinks=true]{hyperref}
\usepackage{tkz-euclide} 
\usepackage{listings}
\usepackage{gvv}      
%\def\inputGnumericTable{}                                 
\usepackage[latin1]{inputenc}                                
\usepackage{color}                                            
\usepackage{array}                                            
\usepackage{longtable}                                       
\usepackage{calc}                                             
\usepackage{multirow}    
\usepackage{hhline}                                           
\usepackage{ifthen}     

\usepackage{lscape}
\usepackage{tabularx}
\usepackage{array}
\usepackage{float}
\usepackage{multicol}

\newtheorem{theorem}{Theorem}[section]
\newtheorem{problem}{Problem}
\newtheorem{proposition}{Proposition}[section]
\newtheorem{lemma}{Lemma}[section]
\newtheorem{corollary}[theorem]{Corollary}
\newtheorem{example}{Example}[section]
\newtheorem{definition}[problem]{Definition}
\newcommand{\BEQA}{\begin{eqnarray}}
\newcommand{\EEQA}{\end{eqnarray}}
\newcommand{\define}{\stackrel{\triangle}{=}}
\theoremstyle{remark}
\newtheorem{rem}{Remark}

% Marks the beginning of the document
\begin{document}
\bibliographystyle{IEEEtran}
\vspace{3cm}

\title{JEE Main 2020 \\ 05-09-2024 Shift2}
\author{ai24btech11014 - Charitha Sri}

\maketitle
\bigskip       
\renewcommand{\thefigure}{\theenumi}
\renewcommand{\thetable}{\theenumi}



\section{Math}                                                      
\begin{enumerate}                     
\item If $x=1$ is a critical point of the function $f\brak{x} = \brak{3x^{2}+ax-2-a}e^{x}$, then:        
\begin{enumerate}                            
\item $x=1$ is a local minima and $x = -\frac{2}{3}$ is a local maxima of $f$.          
\item $x=1$ is a local maxima and $x=-\frac{2}{3}$ is a local minima of $f$.            
\item $x=1$ and $x=-\frac{2}{3}$ are local minima of $f$.           
\item $x=1$ and $x=-\frac{2}{3}$ are local maxima of $f$.       
\end{enumerate}
\item
	  \begin{align}
		  \lim_{x \to 0}\frac{x\brak{{e}^{\frac{\sqrt{1+x^{2}+x^{4}} - 1}{x}} -1 }}{\sqrt{1+x^{2}+x^{4}} - 1}
	  \end{align}
\begin{enumerate}
  \begin{multicols}{4}
  \item  is equal to $\sqrt{e}$
  \item is equal to $1$
  \item is equal to $0$
  \item does not exist
  \end{multicols}
  \end{enumerate}
 \item The statement $(p \rightarrow (q \rightarrow p)) \rightarrow (p \rightarrow (p \cup q))$ is:
  \begin{enumerate}
 \item equivalent to $\brak{p \cup q} \cap \brak{ \sim p}$       
 \item equivalent to $\brak{p \cap q} \cup \brak{ \sim p}$      
 \item a contradiction
  \item a tautology
  \end{enumerate}
  \item If $L = \sin^2\left(\frac{\pi}{16}\right) - \sin^2\left(\frac{\pi}{8}\right)$
 and $M = \cos^2\left(\frac{\pi}{8}\right) - \sin^2\left(\frac{\pi}{8}\right)$, then:                 
 \begin{enumerate}  
		 \begin{multicols}{2}
 \item $M = \frac{1}{2\sqrt{2}}+\frac{1}{2}\cos\frac{\pi}{8}$      
 \item $M = \frac{1}{4\sqrt{2}}+\frac{1}{4}\cos\frac{\pi}{8}$    
 \item $L = -\frac{1}{2\sqrt{2}}+\frac{1}{2}\cos\frac{\pi}{8}$    
		 \item $L=\frac{1}{4\sqrt{2}}-\frac{1}{4}\cos\frac{\pi}{8}$                         \end{multicols}    
 \end{enumerate}
 \item If the sum of the first $20$ terms of the series $\log_{{7}^{\frac{1}{2}}} x + \log_{{7}^{\frac{1}{3}}} x + \log_{{7}^{\frac{1}{4}}} x + \dots$ is $460$, then $x$ is equal to:
 \begin{enumerate}
 \begin{multicols}{4}
 \item ${7}^{\frac{1}{2}}$
 \item ${7}^{2}$
 \item ${e}^{2}$
\item ${7}^{\frac{46}{21}}$
 \end{multicols}
 \end{enumerate} 
 \item There are $3$ sections in a question paper and each section contains $5$ questions candidate has to answer a total of $5$ questions, choosing at least one question from each section. Then the number of ways, in which the candidate can choose the questions, is:
 \begin{enumerate}
 \begin{multicols}{4}
     \item $2250$
     \item $2255$
     \item $1500$
     \item $3000$
\end{multicols}
     \end{enumerate}
     \item If the mean and the standard deviation of the data $3,5,7,a,b$ are $5$ and $2$ respectively, then $a$ and $b$ are the roots of the equation:
     \begin{enumerate}
    \begin{multicols}{4}
\item $x^{2}-20x+18=0$
\item $x^{2}-10x+19=0$
\item $2x^{2}-20+19=0$
\item $x^{2}-10x+18=0$
     \end{multicols}
     \end{enumerate}
     \item The derivative of $\tan^{-1}\brak{\frac{\sqrt{1+x^{2}}-1}{x}}$ with respect to $\tan^{-1}\brak{\frac{2x\sqrt{1-x^{2}}}{1-2x^{2}}}$ at $x=\frac{1}{2}$ is:
     \begin{enumerate}
     \begin{multicols}{4}
     \item $\frac{ 2 \sqrt{3}}{3}$
     \item $\frac{2\sqrt{3}}{5}$
     \item $\frac{\sqrt{3}}{12}$
     \item $\frac{\sqrt{3}}{10}$
     \end{multicols}
      \end{enumerate}
      \item If 
	      \begin{align}
		      \int \frac{\cos\theta}{5+7\sin\theta-2\cos^{2}\theta}d\theta= A\log_{e}|B\brak{\theta}|+C
	      \end{align}
	      where C is a constant of integration, then $\frac{B\brak{\theta}}{A}$ can be:
      \begin{enumerate}
      \begin{multicols}{4}
      \item $\frac{5\brak{2\sin\theta+1}}{\sin\theta+3}$
      \item $\frac{5\brak{\sin\theta+3}}{2\sin\theta+1}$
       \item $\frac{2\sin\theta+1}{\sin\theta+3}$
       \item $\frac{2\sin\theta+1}{5\brak{\sin\theta+3}}$
      \end{multicols}
      \end{enumerate}
      \item If the length of the cord of the circle, $x^{2}+y^{2}=r^{2}\brak{r>0}$ along the line, $y-2x=3$ is $r$, then $r^{2}$ is equal to:
      \begin{enumerate}
      \begin{multicols}{4}
      \item $12$
      \item $\frac{24}{5}$
      \item $\frac{9}{5}$
      \item $\frac{12}{5}$
      \end{multicols}
      \end{enumerate}
      \item If $\alpha$ and $\beta$ are the roots of the equation, $7x^{2}-3x-2=0$, then the value of$\frac{\alpha}{1-\alpha^{2}} +\frac{\beta}{1-\beta^{2}}$
      \begin{enumerate}
      \begin{multicols}{4}
      \item $\frac{27}{32}$
      \item $\frac{1}{24}$
      \item $\frac{27}{16}$
      \item $\frac{3}{8}$
      \end{multicols}
      \end{enumerate}
      \item If the sum of the second, third and fourth terms of a positive term G.P. is $3$ and the sum of its sixth, seventh and eighth terms is $243$, then the sum of the first $50$ terms of the G.P. is:
       \begin{enumerate}
      \begin{multicols}{4}
      \item $\frac{2}{13}\brak{3^{50}-1}$
      \item $\frac{1}{26}\brak{3^{49}-1}$
      \item $\frac{1}{13}\brak{3^{50}-1}$\item $\frac{1}{26}\brak{3^{50}-1}$
      \end{multicols}
      \end{enumerate}
     \item If the line $y=mx+c$ is a common tangent to the hyperbola $\frac{x^{2}}{100} - \frac{y^{2}}{64}=1$ and the circle $x^{2}+y^{2}=36$, then which one of the following is true?
      \begin{enumerate}
      \begin{multicols}{4}
      \item $4c^{2}=369$
      \item $c^{2}=369$
      \item $8m+5=0$
      \item $5m=4$
      \end{multicols}
      \end{enumerate}
      \item The area (in sq.units) of the region 
	      \begin{align}
		      A = \cbrak{\brak{x,y}:\brak{x-1} \sbrak{x} \leq y \leq 2\sqrt{x}, 0\leq x\leq 2}
	      \end{align}
	      where $\sbrak{t}$ denotes the greatest integer funtion, is:
      \begin{enumerate}
      \begin{multicols}{4}
      \item $\frac{4}{3}\sqrt{2}-\frac{1}{2}$
      \item $\frac{8}{3}\sqrt{2}-\frac{1}{2}$
      \item $\frac{8}{3}\sqrt{2}-1$
      \item $\frac{4}{3}\sqrt{2}+1$
      \end{multicols}
      \end{enumerate}
     \item If $a+x=b+y=c+z+1$, where $a, b, c, x, y, z$ are non-zero distinct real numbers, then  
    $\left| \begin{matrix} x & a+y & x+a \\ y & b+y & y+b \\z & c+y & z+c  \end{matrix} \right|$ is equal to:
     \begin{enumerate}
      \begin{multicols}{4}
      \item $y\brak{a-b}$
      \item $0$
      \item $y\brak{b-a}$
      \item $y\brak{a-c}$
       \end{multicols}
      \end{enumerate}

      
 \end{enumerate}
 \end{document}
