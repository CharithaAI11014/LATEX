%iffalse
\let\negmedspace\undefined
\let\negthickspace\undefined
\documentclass[journal,12pt,onecolumn]{IEEEtran}
\usepackage{cite}
\usepackage{amsmath,amssymb,amsfonts,amsthm}
\usepackage{algorithmic}
\usepackage{graphicx}
\usepackage{textcomp}
\usepackage{xcolor}
\usepackage{txfonts}
\usepackage{listings}
\usepackage{enumitem}
\usepackage{mathtools}
\usepackage{gensymb}
\usepackage{comment}
\usepackage[breaklinks=true]{hyperref}
\usepackage{tkz-euclide} 
\usepackage{listings}
\usepackage{gvv}      
%\def\inputGnumericTable{}                                 
\usepackage[latin1]{inputenc}                                
\usepackage{color}                                            
\usepackage{array}                                            
\usepackage{longtable}                                       
\usepackage{calc}                                             
\usepackage{multirow}    
\usepackage{hhline}                                           
\usepackage{ifthen}     

\usepackage{lscape}
\usepackage{tabularx}
\usepackage{array}
\usepackage{float}
\usepackage{multicol}

\newtheorem{theorem}{Theorem}[section]
\newtheorem{problem}{Problem}
\newtheorem{proposition}{Proposition}[section]
\newtheorem{lemma}{Lemma}[section]
\newtheorem{corollary}[theorem]{Corollary}
\newtheorem{example}{Example}[section]
\newtheorem{definition}[problem]{Definition}
\newcommand{\BEQA}{\begin{eqnarray}}
\newcommand{\EEQA}{\end{eqnarray}}
\newcommand{\define}{\stackrel{\triangle}{=}}
\theoremstyle{remark}
\newtheorem{rem}{Remark}

% Marks the beginning of the document
\begin{document}
\bibliographystyle{IEEEtran}
\vspace{3cm}

\title{22.MISCELLANEOUS}
\author{ai24btech11014 - Charitha Sri}

\maketitle
\bigskip       
\renewcommand{\thefigure}{\theenumi}
\renewcommand{\thetable}{\theenumi}

\section{Section A}

\begin{enumerate}
	
	\item A variable takes value x with frequency ${}^{n+x-1}C_{x},
		x=0,1,2,\dots n$. The mode of the variable is $\dots$
		\hfill{\brak{1982 - 2 Marks}}
\section{Section B}
	\item For real numbers x and y, we write x*y if $x-y+\sqrt{2}$ is an irrational number. Then, the relation * is an equivalence relation.
		\hfill{\brak{1981 - 2 Marks}}
\section{Section C}
	\item If $X$ and $Y$ are two sets, then $X\cap \brak{X\cup Y}^c$ equals.
		\hfill{\brak{1979}}
		\begin{enumerate}
		\begin{multicols}{2}
		\item X
		\item Y
                \item $\phi$
	        \item None of these
		\end{multicols}
	        \end{enumerate}
	\item The expression $\frac{12}{3+\sqrt{5}+2\sqrt{2}}$ is equal to 
		\hfill{\brak{1980}}
		\begin{enumerate}
                \item $1-\sqrt{5}+\sqrt{2}+\sqrt{10}$
		\item $1+\sqrt{5}+\sqrt{2}-\sqrt{10}$
		\item $1+\sqrt{5}-\sqrt{2}+\sqrt{10}$
		\item $1-\sqrt{5}-\sqrt{2}+\sqrt{10}$
		\end{enumerate}
	\item Select the correct alternative in each of the following. Indicate your choice by the appropriate letter only.Let S be the standard deviation of n observations. Each of the n observations is multiplied by a constant c. Then the standard deviation of the resulting number is \hfill{\brak{1980}}
		\begin{enumerate}
		\begin{multicols}{2}
		\item s
		\item cs
		\item s$\sqrt{c}$
		\item none of these
		\end{multicols}
		\end{enumerate}
	\item The standard deviation of 17 numbers is zero. Then \hfill{(1980)}
		\begin{enumerate}
		\item The numbers are in geometric progression with common ratio not equal to one.
		\item Eight numbers are positive, eight are negative and one is zero.
		\item either (a) or (b)
		\item none of these
		\end{enumerate}
	\item Consider any set of 201 observations $x_{1},x_{2},\dots x_{200},x_{201}$. It is given that $x_{1}<x_{2}<\dots <x_{200}<x_{201}$.Then the mean deviation of this set of observations about a point k is minimum when k equals \hfill{(1981 -2 Marks)}
		\begin{enumerate}
			\item $\brak{x_{1}+x_{2}+\dots+x_{200}+x_{201}}/201$
		\item $x_{1}$
		\item $x_{101}$
		\item $x_{201}$
		\end{enumerate}

	\item If $x_{1},x_{2},\dots,x_{n}$ are any ral numbers and n is any positive integer, then \hfill{(1982 - 2 Marks)}
		\begin{enumerate}
			\item $n\sum_{i=1}^{n} {x_{i}}^{2}<\brak{\sum_{i=1}^{n} x_{i}}^{2}$
			\item $\sum_{i=1}^{n} {x_{i}}^{2} \geq \brak{\sum_{i=1}^{n} x_{i}}^{2}$
			\item $\sum_{i=1}^{n} {x_{i}}^{2} \geq n \brak{\sum_{i=1}^{n} x_{i}}^{2}$
		\item none of these
		\end{enumerate}

	\item Let $S={1,2,3,4}$.The total number of unordered pairs of disjoint subsets of S is equal to \hfill{\brak{2010}}
		\begin{enumerate}
                \begin{multicols}{4}
		\item $25$
		\item $34$
		\item $42$
		\item $41$
		\end{multicols}
		\end{enumerate}

	\item Let $P={\theta:\sin\theta-\cos\theta=\sqrt{2} \cos\theta}$ and $Q={\theta:\sin\theta+\cos\theta=\sqrt{2}\sin\theta}$ be two sets. Then \hfill{\brak{2011}}
		\begin{enumerate}
		\begin{multicols}{2}
		\item $P \subset Q and Q-P \neq \varnothing$
		\item $Q \not\subset P$
		\item $P \not\subset Q$
		\item $P=Q$
		\end{multicols}
		\end{enumerate}
\section{D}
	\item In a college of 300 students every student reads 5 newspapers and every newspaperis read by 60 students.The number of newspapers is \hfill{\brak{1998 -2 Marks}}
		\begin{enumerate}
		\item at least $30$
		\item at most $20$
		\item exactly $25$
		\item none of these
		\end{enumerate}

	\item Let $S_{1},S_{2},\dots$ be squares such that for each $n \geq 1$, the length of a side of $S_{n}$ equals the length of a diagonal of $S_{n+1}$.If the length of a side of $S_{1}$ is 10cm, then for which of the following values of n is the area of $S_{n}$ less than 1sq.cm? \hfill{\brak{1999 - 3 Marks}}
		\begin{enumerate}
		\begin{multicols}{4}
		\item $7$
		\item $8$
		\item $9$
		\item $10$
		\end{multicols}
		\end{enumerate}

	\item Let $S ={1,2,3,\dots ,9}$. For $k=1,2,\dots,5,$ let $N_{k}$ be the number of subsets of S, each containing five elements out of which exactly k are odd. Then $N_{1}+N_{2}+N_{3}+N_{4}+N_{5}=$ \hfill{\brak{Jee Adv. 2017}}
		\begin{enumerate}
		\begin{multicols}{4}
		\item $210$
		\item $252$
		\item $125$
		\item $126$
		\end{multicols}
		\end{enumerate}
\section{E}
	\item An investigator interviewed 100 students to determine their preferences for the three drinks : milk (M), coffee (C) and tea (T).He reported the following :10 students had all the three drinks M,C and T;20 had M and C; 30 had C and T;25 had M and T; 12 had M only ; 5 had C only ; and 8 had T only.Using a Venn diagram find how many did not take any of the three drinks. \hfill{\brak{1978}}
	\item (a) Construct a triangle with base 9cm and altitude 4cm, the ratio of the other two sides being 2:1
		
		(b) Construct a triangle in which the sum of the three sides is 15cm with base angles $60^\circ$ and $45^\circ$. Justify your steps. \hfill{\brak{1979}}
\end{enumerate}

\end{document}
